\documentclass[pdflatex,sn-mathphys]{sn-jnl}% Math and Physical Sciences Reference Style

%%\jyear{2021}%
\raggedbottom
%%\unnumbered% uncomment this for unnumbered level heads

\begin{document}

\title{Macroeconomic Populism in the 21st Century: Revisiting Dornbusch \& Edwards}


%%=============================================================%%
%% Prefix	-> \pfx{Dr}
%% GivenName	-> \fnm{Joergen W.}
%% Particle	-> \spfx{van der} -> surname prefix
%% FamilyName	-> \sur{Ploeg}
%% Suffix	-> \sfx{IV}
%% NatureName	-> \tanm{Poet Laureate} -> Title after name
%% Degrees	-> \dgr{MSc, PhD}
%% \author*[1,2]{\pfx{Dr} \fnm{Joergen W.} \spfx{van der} \sur{Ploeg} \sfx{IV} \tanm{Poet Laureate} 
%%                 \dgr{MSc, PhD}}\email{iauthor@gmail.com}
%%=============================================================%%

\author[1]{\fnm{Nicolas} \sur{Cachanosky}}\email{ncachano@msudenver.edu}

\author[2,3]{\fnm{João Pedro} \sur{Bastos}}\email{Joao-Pedro.Bastos@ttu.edu}

\author[4]{\fnm{Tomas} \sur{Faintich}}\email{tomifain@gmail.com
}

\affil[1]{\orgdiv{Department of Economics, Metropolitan State University of Denver},\orgaddress{\city{Denver}, \state{CO}, \country{United States}}}

\affil[2]{\orgdiv{Department of Agricultural and Applied Economics}, 

\orgdiv{Free Market Institute}, 
\orgname{Texas Tech University}, \orgaddress{\city{Lubbock}, \state{TX}, \country{United States}}}

\affil[3]{\orgdiv{} \orgname{Universidad del CEMA}, \orgaddress{\city{Buenos Aires}, \country{Argentina}}}





%%==================================%%
%% sample for unstructured abstract %%
%%==================================%%

\abstract{The seminal work of Dornbusch and Edwards (1990) described that the mechanism through which populist leaders stepped into power and implemented their policies in Latin American had common features along different countries. Their work discusses the similarities of populist governments in Peru and Chile. We replicate their findings for the 21st century Latin America, expanding }

\keywords{Latin America, Populism, Inflation.}

%%\pacs[JEL Classification]{D8, H51}

%%\pacs[MSC Classification]{35A01, 65L10, 65L12, 65L20, 65L70}
\maketitle

\begin{center}
    (Version as of: January 2021)
\end{center}
\bigskip


\newpage
\section{Introduction}\label{sec1}

The seminal work of Dornbusch \& Edwards (1990) advanced the idea that while different in many aspects – ideology included – populism governments have a common economic agenda. As they called, "macroeconomic populism" is defined by governments that pursued growth and income distribution at all costs, neglecting the negative effects of deficits and inflation. 

Their work provided two case studies, that of Chile under the Unidad Popular (UP) of Salvador Allende (1970-73), and Peru under Alán Garcia (1985-1990). Indeed, while notoriously different these two governments shared "critical economic factors". Starting amid under-performing economic growth, both proposed a "reactivation" of the economy, redistribution of income, and restructuring the industry. These happened in a similar fashion in both countries.  

Since the economy was under-performing, spare capacity in the industry should, in theory, allow for a "safe" monetary expansion, without inflationary pressure. Real wages went up, promoting redistribution of income. The economy, at first, took off. But such growth is unsustainable. The demand shock is not accompanied by a real increase in production. Imports fall as international reserves diminish, impacting both supply of final goods and of inputs for the domestic industry. Inflation rises fast and shortages occur everywhere. The situation quickly deteriorates, sharply reducing income, and chaos is installed.

Looking at Argentina under the Kirchners, Bolivia under Evo Morales, the Ecuador of Rafael Correa, Daniel Ortega's Nicarágua, and the "Bolivarian" Venezuela of Chavez and Maduro, we analyze the macroeconomic populism thesis for the twenty-first Latin America. 

Section 2 discusses Dornbusch & Edwards's theory in detail. Section 3 discusses our selection of countries and the policies they implemented. Section N concludes. 

\smallskip
\section{The "Macroeconomic Populism" Thesis}\label{sec2}
\smallskip

Political science has been struggling to provide a consistent definition of populism. Rather then a specific type of government with a strict set of policies, the field has provided us with a bundle of policies and characteristics that most populist governments have. At the time, Dornbusch & Edwards (1990) relied mostly in the work of Drake (1982) which proposes that populist are characterized by 1) the usage of "political mobilization, recurrent rhetoric and symbols designed to inspire the people"; 2) the formation of a heterogeneous coalition that while includes and is led by the middle class and even part of elites, is aimed to advance the interests of the working class; this is done by (3) a reformist agenda – usually expanding the role of the state to accelerate industrialization – to promote development allied with redistributive policies.

Since then, we may highlight the work of de la Torre (2016, 2017), Weyland (2001,2009), Doyle (2011)  and that of Edwards (2010, 2019) himself\footnote{See also Kaltwasser \textit{et al.} (2017) for a comprehensive review of the subject.}. We also should mention works that try to \textit{measure} populism such as Hawkins (2009), Ramos (2018). 

\smallskip
Populists usually are charismatic individuals, who lead a personalistic government. They rely on a "us versus them" rhetoric, where the populist leader "saves" the people – conceived as an unique and homogeneous body\footnote{Though, see de la Torre (2016) for examples of "the People" conceived as heterogeneous.} – from the abuses of a ruling elite. In opposition to the rule of law, "The People" is also the ultimate source of legitimacy (Abts & Rummens, 2007). This narrative also has a nationalistic component (de la Torre, 2017a), given that the elites are generally comprised by large multinational corporations, and organisations such as the International Monetary Fund, notoriously.

\smallskip
To use Dornbusch & Edwards (1990, p.247) own definition: "Macroeconomic populism is an approach to economics that emphasizes growth and income distribution and deemphasizes the risks of inflation and deficit finance, external constraints and the reaction of economic agents to aggressive non-market policies". Put simply, populism is a violation of the "good economics" of fiscal responsibility, budget constraints, and efficiency (Edwards, 2019). 

\smallskip
Hence, populist macroeconomics starts with a common diagnosis of the country's economic issues. These are the three \textit{R}'s of populist macroeconomics: \textit{reactivate, redistribute, restructure}. Although not necessary, the initial conditions comprises an unequal society claiming for political change, since stagnation or insufficient growth is maintaining standards of living at undesirable levels. Previous stabilization attempts, often guided by a program elaborated by the IMF, may have given some leeway for expansionist policies. 

\smallskip
Crucially, then, is the understanding that the country has a hiatus on production, i.e. there is spare capacity in the industry that should allow for aggressive expansionist policies without the dangers of inflation. If the disease is an idle industry, the solution is to first, \textit{reactivate} it. Aligned with the expansionist policies is the \textit{redistribution} of income, through significant increases in real wages\footnote{Though Dornbusch \& Edwards (1990) do not point out to this, the redistribution of income has two purposes. The economic rationale is that lower income brackets employ proportionally more of their income into spending, while the elites are only "extracting rents" – \textit{los rentistas} ("the rentists") are culprits that can be found in every crime scene of Latin American politics. The second related purpose is to appeal for the political narrative of "us versus them", of inequality and exploitation of the people by the elites.}. Finally, \textit{restructure} the domestic economy, especially by the industrialization of the economy\footnote{However, de la Torre (2017b, p. 196) argues that industrialization was mainly intended for relatively more developed nations as Brazil, Argentina, and Mexico, not being found in populist policies in Bolivia, Ecuador and Peru.}. The way to achieve industrialization, as promoted by the United Nations Economic Commission for Latin America and the Caribbean ("Económica para América Latina y el Caribe", CEPAL), was through subsidies to the industry and import substitution industrialization policies\footnote{These ideas were famously advocated by scholars such as Raúl Prebisch and Celso Furtado. For a discussion of these ideas, see Aguilar (1986). See Prebisch (e.g. 1976;1987). Also see Ocampo (2001) for a complete review of Prebisch's ideas. For Furtado, see his \textit{Desenvolvimento e Subdesenvolvimento} (1961 [2000]).}.

\smallskip
Following, Dornbusch & Edwards argue that the implementation of populist policies generally ends up having four stages, or steps. 

\smallskip
\textit{1.} In the beginning, the macroeconomic scenario responds successfully. Expansion leads to growing output, real wages and employment also follow. Since there are inventories to be drawn and reserves to be used to finance imports, there is no danger of shortages and inflation. 

\smallskip
\textit{2.} Not long afterwards (around a year later), as foreign reserves start to reach critical levels, imports  are reduced, affecting both final and capital goods, the latter also impacting the domestic industry with lacking inputs for production. Bottlenecks start to appear everywhere; shortages become common and inflation rises fast. Exchange rate manipulation, subsidies and price controls are attempted to stabilize the economy. 

\smallskip
\textit{3.} As wages are being constantly increase to keep up with inflation, massive subsidies become necessary to boost domestic production, deteriorating the fiscal situation of the country, with increasing budget deficits. Capital flights away and the economy collapses. Chaotic shortages result, accompanied by severe deficits, since tax collection declines with decreasing economic activity. When the economy finally collapses, real wages fall rapidly, as subsidies are cut and the economy shuts down.

\smallskip
\textit{4.} A new government assumes and implements orthodox policies for stabilization, probably an IMF program. With a strong retraction, real wages sharply fall, and the country is at an income level lower than before. Since capital has flown overseas, the impact is enduring, and the economy struggles to rebuild. The comeback depends on local politics allowing for the adjustment process; the credibility of the new government will dictate if foreign investments will be attracted to rebuild.

\section{Populism in the 21st Century}\label{sec2}

We investigate if and exactly how this thesis applies to 21st populism in Latin America. Even though there is no unique definition of populism, there is a fair consensus of what countries that had populism governments\footnote{See (e.g.) Absher, Grier, & Grier (2020), Colburn & Cruz (2012) de la Torre (2013, 2017b), Ocampo (year), Bittencourt (2012), Doyle (2011). While Brazil under the Worker's Party of Lula (2003-10) and Dilma (2011-16) are potential candidates, especially regarding their \textit{policies}, they are significantly less mentioned in the literature}.

\textbf{Talk about "Neo-Populism"?}

Based on the literature, we select 5 populist governments\footnote{For another discussion of populism in these countries, see Cachanosky & Padilla (2019).}. Argentina under the Kirchner's, that is, Néstor Kirchner (2003-07) and his wife Cristina Kirchner (2007-15); the government of Evo Morales (2006-2019) in Bolivia; Ecuador under the presidency of Rafael Correa (2007-2016); the Nicaragua of Daniel Ortega (2007-present) and the Bolivarian Venezuela of Hugo Chávez (1999-2013) and Nicolás Maduro (2013-2019)\footnote{On January 10, 2019, the National Assembly declared the 2018 presidential elections as invalid due to fraud, and its president, Juan Guaidó assumed as interim president. The presidency has been disputed so far}. 

\subsection{Argentina}

Néstor and Cristina Kirchner both were elected through the policial party Partido Justicialista (PJ). Founded in 1947 by Juan Perón, one of the fathers of Latin America "golden age" of populism (de la Torre, 2017b)\footnote{See also Ocampo (2015a, 2015b)}. The Peronists were in charge of the presidency for 37 out 75 years since then; they received between 30 and over 60\% of votes in presidential and parliamentary elections, while also governing multiple provinces and municipalities around the country.

Néstor assumes on May 25rd, 2003, amid a political and economic crisis, where five presidents (four of them from the Partido Judicilista, PJ) have taken office in less then 3 years.\footnote{In December 20th, 2001, Fernando de la Rúa (UCR –Unión Cívica Radical- Party) resigns. Ramón Puerta (PJ) assumes for a period of two days, being replaced by Adolfo Rodriguez Saá (PJ) on December 22nd, who declares default. A week later, on December 30th, he is forced out of office. Finally, in January 2nd, Congress appoints Eduardo Duhalde (PJ) as the President until de la Rua's original term was supposed to end, when new elections were called – Néstor wins}. When he steps in, the budget has a surplus, due to the default, tax increases, and currency depreciation, but the situation quickly deteriorates.



\subsection{Bolivia}

Evo Morales was the Bolivian president for three consecutive from 2006 to 2019, when he resigned. Bolivia's economic performance during his initial mandate was praised by some authors as "remarkable", in part due to a "large-scale and well-timed increase in public spending" (see Weisbrot, Ray, & Johnston, 2009, p. 6). 



\subsection{Ecuador}

\subsection{Nicaragua}

\subsection{Venezuela}

Venezuela is indisputably the flagship example of Latin American populism.



Populist policies reduce economic freedom, namely through increased regulation, reduced trade, and less secured property rights. 



\section{Discussion}\label{sec3}

\smallskip


\section*{References}


\smallskip

Absher, Samuel, Kevin Grier & Robin Grier. (2020). "The economic consequences of durable left-populist regimes in Latin America". \textit{Journal of Economic Behavior and Organization} 177: 787-817. 

\smallskip
Aguilar, Renato. (1986) "Latin American structuralism and exogenous factors", \textit{Social Science Information}  25(1), pp. 277–290.

\smallskip
Cachanosky, Nicolás. (2018). "The Cost of Populism in Argentina, 2003-2015", \textit{MISES: Interdisciplinary Journal of Philosophy, Law, and Economics.}

\smallskip
Cachanosky, Nicolás, and Alexandre Padilla (2019). "Latin American Populism in the Twenty-First Century", \textit{The Independent Review} 24(2): 209-226.

\smallskip
Caldentey, Esteban P., Osvaldo Sunkel, and Miguel T. Olivos. (2012). \textit{Raúl Prebisch (1901-1986): Un recorrido por las etapas de su pensamiento sobre el desarrollo económico}. Santiago, Chile: Naciones Unidas, CEPAL.

\smallskip
Colburn, Forrest D., and Arturo Cruz S. (2012)."Personalism and Populism in Nicaragua." \textit{Journal of Democracy} 23(2): 104-118

\smallskip
de la Torre, Carlos. (2013). "In the Name of the People: Democratization, Popular Organizations, and Populism in Venezuela, Bolivia, and Ecuador." \textit{European Review of Latin American and Caribbean Studies} 95: 27–48.

\smallskip
de la Torre, Carlos. (2016). "Populism and the politics of the extraordinary in Latin America", Journal of Political Ideologies, 21:2, 121-139,

\smallskip
de la Torre, Carlos (2017a). "Populism and Nationalism in Latin America", \textit{Javnost - The Public} 24(4): 375-390.

\smallskip
de la Torre, Carlos (2017b). "Populism in Latin America", in: \textit{The Oxford Handbook of Populism}. Kaltwasser, C. R., Taggart, P. A., Espejo, P. O., & Ostiguy, P. (Eds.). Oxford: Oxford University Press.

\smallskip
Dornbusch, Rudiger, and Sebastian Edwards. (1990). "Macroeconomic Populism", \textit{Journal of Development Economics} 32(2): 247-277

\smallskip
Dornbusch, Rudiger, and Sebastian Edwards. 1991. \textit{The Macroeconomics of Populism in Latin America.} Chicago: University of Chicago Press

\smallskip
Doyle, David. (2011) ‘The Legitimacy of Political Institutions: Explaining Contemporary Populism in Latin America’, Comparative Political Studies, 44(11), pp. 1447–1473

\smallskip
Drake, P. (1982) "Conclusion: Requiem for populism? in: \textit{Latin American Populism in Comparative Perspective}, ed. M. L. Conniff. Albuquerque, NM: University of New Mexico Press.

\smallskip
Edwards, Sebastian. (2010). \textit{Left Behind: Latin America and the false promise of populism}. Chicago: Chicago University Press.

\smallskip
Edwards, Sebatian. (2019). "On Latin American Populism, and Its Echoes around the World"\textit{Journal of Economic Perspectives},  33(4): 76–99

\smallskip
Furtado, Celso. (1961 [2000]). Desenvolvimento e subdesenvolvimento. In: \textit{Cinqüenta anos de pensamento na CEPAL}, pp. 239-262.Rio de Janeiro: Record/CEPAL.

\smallskip
Hawkins, K. A. (2009) "Is Chávez populist? Measuring populist discourse in comparative perspective" \textit{Comparative Political Studies} 42(8): 1040-1067

\smallskip
Kaltwasser, C. R., Taggart, P. A., Espejo, P. O., & Ostiguy, P. (Eds.). (2017). \textit{The Oxford handbook of Populism}. Oxford University Press.

\smallskip
Knight, A. (1998). Populism and neopopulism in Latin America, especially Mexico. \textit{Journal of Latin American Studies}, 30, 223-248. 

\smallskip
Ocampo, E. (2015a). \textit{Entrampados en la farsa: El populismo y la decadencia argentina}. Buenos Aires: Claridad.

\smallskip
Ocampo, E. (2015b). \textit{Commodity Price booms and populist cycles. An Explanation of Argentina's Decline in the 20th Century}. (Documentos de Trabajo Nº 562). Buenos Aires.

\smallskip
Ocampo, José Antonio. (2001). "Raúl Prebisch and the development agenda at the dawn of the twenty-first century", \textit{CEPAL Review}, 75: 23-37.

\smallskip
Prebisch, Raúl. (1976). "Crítica al capitalismo periférico", \textit{Revista de la CEPAL}, 7-74.

\smallskip
Prebisch, Raúl. (1987). "Cinco etapas de mi pensamiento sobre el desarrollo", In: \textit{Raúl Prebisch: un aporte al estudio de su pensamiento}. Santiago, Chile: CEPAL.

\smallskip
Ramos, Julián Martínez. (2018) "Populist discourse and civic culture: insights from Latin America"\textit{Política / Revista de Ciencia Política} 56(2): 7-23.

\smallskip
Rode, M., Revuelta, J. (2015). "The Wild Bunch! An empirical note on populism and economic institutions. \textit{Economics of Governance} 16: 73–96 

\smallskip
Weisbrot, M., Ray, R., Johnston, J., 2009. Bolivia: The Economy During the Morales Administration (No. 2009-47). Center for Economic and Policy Research
(CEPR).

Weyland, Kurt. (2001). "Clarifying a contested concept: Populism in the study of Latin American politics." \textit{Comparative Politics}, 34, 1-22. 

Weyland, Kurt. (2009). "The rise of Latin America’s two lefts? Insights from rentier state theory." \textit{Comparative Politics}, 41: 145-164. 





\smallskip

\newpage
\section*{Table Templates}

\begin{table}[h!]
\begin{center}
\begin{minipage}{\textwidth}
\caption{Title}\label{tab2}
\begin{tabular*}{\textwidth}{@{\extracolsep{\fill}}lcccccccc@{\extracolsep{\fill}}}
\toprule% 
Variable&&
\multicolumn{2}{@{}c@{}}{$Variable 1$} & 
\multicolumn{2}{@{}c@{}}{$Variable 2$}&
\multicolumn{2}{@{}c@{}}{$Variable 3$ }\\
\cmidrule{3-4}\cmidrule{5-6}\cmidrule{7-8}%
& Var & Mean & Std & Mean & Std & Mean & Std \\
\midrule
Variable & & 2.598 & 1.406 & 1.910 & 8.038 & 11.853 & 16.891\\
  &  Male & 2.622 & 1.400 & 1.980 & 8.114 & 12.031 & 16.711\\
   &  Female & 2.642 & 1.394 & 2.033 & 8.502 & 11.445 & 17.304\\
   \\

\botrule
\end{tabular*}
\footnotetext{Notes: }
\end{minipage}
\end{center}
\end{table}

\begin{table}[h!]
\begin{center}
\begin{minipage}{\textwidth}
\caption{Title}\label{tab2}
\begin{tabular*}{\textwidth}{@{\extracolsep{\fill}}lcccccccc@{\extracolsep{\fill}}}
\toprule
& 1 & 2 & 3  & 4  & 5\\
& (1) & (2) & (3) & (4) & (5)\\ 
\midrule
$Sample$    & 0.432   & 0.993  & 1.216 & 1.431 & 1.745  \\
$Sample$  & 8.789  & 8.500  & 8.737 & 8.706 & 8.593  \\
\botrule
\end{tabular*}
\end{minipage}
\end{center}
\end{table}


\newpage
\begin{table}[!ht]
\section*{Appendix 2 }
\smallskip
\subsubsection*{Table 10: Aggregated Stratification Sector Classifications}
\begin{tabular}{|l|p{19pc}|}
\hline
\textbf{Grouped} & \textbf{Original}
\\\hline
Services & Services; Other Services; Services of Motor Vehicles; Other Services Panel; IT \& IT Services
\\\hline
Retail \& Wholesale (R\&W) & Retail; Retail Panel; Wholesale; Wholesale \& Retail; Wholesale of Agri Inputs \& Equipament; Services of Motor Vehicles/Wholesale/Retail 
\\\hline
Manufacturing & Manufacturing; Other Manufacturing; Garments; Wood Products; Manufacturing Panel; Leather Products; Furniture; Wood Products \& Furniture; Food; Fabricated Metal Products;  Textiles; Textiles \& Garments; Machinery \& Equipament; Rubber \& Plastic Equipment; Basic Metals/Fabricated Metals/Machinery \& Equip.; Motor Vehicles; Wood products; Furniture; Paper \& Publishing; Machinery \& Equipment; Electronics \& Vehicles; Motor Vehicles \& Transport Equip.; Printing \& Publishing; Electronics; Electronics \& Communications Equip.; Metals, Machinery, Computer \& Electronics
\\\hline
Mining \& Petrochemical & Mining Related Manufacturing; Non-Metallic Mineral Products; Chemicals \& Chemical Products; Chemicals, Plastics \& Rubber; Petroleum products, Plastics \& Rubber
\\\hline
Food \& Hospitality & Hotels \& Restaurants; Hospitality \& Tourism 
\\\hline
Logistics & Transport, Storage, \& Communications; Transport. 
\\\hline
\end{tabular}
\end{table}



\end{document}
