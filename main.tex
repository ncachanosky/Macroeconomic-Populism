% ================================================================
% ================================================================
% --- PREAMBLE
% --- Document class
\documentclass[12pt, letterpaper]{article}

% --- Packages
\usepackage[utf8]{inputenc}
\usepackage[top=1in, bottom=1in, left=1in, right=1in]{geometry}
\usepackage{setspace}
\usepackage{microtype}
\usepackage[dvipsnames]{xcolor}
\usepackage{lastpage}
\usepackage{hyperref}
\usepackage{fancyhdr}
\usepackage{booktabs}
\usepackage{graphicx}
\usepackage{todonotes}
\usepackage{fontspec}
\usepackage{adjustbox}
\usepackage{float}
\usepackage{dcolumn}
\usepackage[backend=biber, style=apa, citestyle=apa]{biblatex}

% --- Settings
\usepackage{libertine}
\usepackage[libertine]{newtxmath}
\addbibresource{references.bib}

\hypersetup{
    colorlinks = true,
    linkcolor  = blue,
    urlcolor   = blue,
    citecolor  = blue,
    }
    
\pagestyle{fancy}

% --- Header and footer
\fancyhf{}
\headheight = 28pt
\chead{\small{Macroeconomic Populism in the 21st Century: Revisiting Dornbusch \& Edwards\\
       \textit{N. Cachanosky, J. P. Bastos, T. Faintich}}}
\cfoot{\footnotesize Page \thepage \hspace{0.5pt} of \pageref{LastPage}} 

% --- Title page
\title{Macroeconomic Populism in the 21st Century: Revisiting Dornbusch \& Edwards
       \bigskip}

\author{
        \textbf{Nicolás Cachanosky} \\
        Metropolitan State University of Denver \\
        Department of Economics \\
        \href{mailto:ncachano@msudenver.edu}{ncachano@msudenver.edu} \\
        \and
        \textbf{João Pedro Bastos} \\
        Texas Tech University \\
        Free Market Institute \\
        \href{mailto:joao-pedro.bastos@ttu.edu}{joao-pedro.bastos@ttu.edu} \\
        \and
        \textbf{Tomás Faintich} \\
        Universidad del CEMA \\
        Friedman-Hayek Center for the Study of a Free Society \\
        \href{mailto:tfaintich22@ucema.edu.ar}{tfaintich22@ucema.edu.ar}
        \bigskip}

\date{\today}


% ================================================================
% ================================================================
% --- DOCUMENT
\setlength{\marginparwidth}{2cm}
\begin{document}


% ================================================================
% --- TITLE PAGE

\maketitle

\begin{abstract}
\noindent
The seminal work of Dornbusch \& Edwards \parencite*{Dornbusch1990} described that the mechanism through which populist leaders stepped into power and implemented their policies in Latin American had common features along different countries. Their work discusses the similarities of populist governments in Peru and Chile. We replicate their findings for the 21st century Latin America, discussing the cases of Argentina, Bolivia, Ecuador, Nicaragua, and Venezuela.
\end{abstract}

\bigskip \bigskip
\footnotesize \noindent \textbf{JEL codes}: TBD \\
\footnotesize \noindent \textbf{Keywords}: Latin America, Populism, Inflation.

\newpage
\doublespacing

% ================================================================
% --- SECTION 1: INTRODUCTION
\section{Introduction} 
    \label{sec:intro}

The seminal work of Dornbusch \& Edwards \parencite*{Dornbusch1990} advanced the idea that while different in many aspects – ideology included – populism governments have a common economic agenda. As they called, "macroeconomic populism" is defined by governments that pursued growth and income distribution at all costs, neglecting the negative effects of deficits and inflation. 

Their work provided two case studies, that of Chile under the Unidad Popular (UP) of Salvador Allende (1970-73), and Peru under Alán Garcia (1985-1990). Indeed, while notoriously different these two governments shared "critical economic factors". Starting amid under-performing economic growth, both proposed a "reactivation" of the economy, redistribution of income, and restructuring the industry. These happened in a similar fashion in both countries.  

Since the economy was under-performing, spare capacity in the industry should, in theory, allow for a "safe" monetary expansion, without inflationary pressure. Real wages went up, promoting redistribution of income. The economy, at first, took off. But such growth is unsustainable. The demand shock is not accompanied by a real increase in production. Imports fall as international reserves diminish, impacting both supply of final goods and of inputs for the domestic industry. Inflation rises fast and shortages occur everywhere. The situation quickly deteriorates, sharply reducing income, and chaos is installed.

Looking at Argentina under the Kirchners, Bolivia under Evo Morales, the Ecuador of Rafael Correa, Daniel Ortega's Nicaragua, and the "Bolivarian" Venezuela of Chavez and Maduro, we analyze the macroeconomic populism thesis for the twenty-first Latin America. 

Section 2 discusses Dornbusch \& Edwards's theory in detail. Section 3 discusses our selection of countries and the policies they implemented. Section N concludes. 


% ================================================================
% --- SECTION 2: MACROECONOMIC POPULISM THESIS
\section{The "Macroeconomic Populism" Thesis} 
    \label{sec:section_2}
    
Political science has been struggling to provide a consistent definition of populism. Rather then a specific type of government with a strict set of policies, the field has provided us with a bundle of policies and characteristics that most populist governments have. At the time, Dornbusch \& Edwards \parencite*{Dornbusch1990} relied mostly in the work of Drake \parencite*{Drake1982} which proposes that populist are characterized by 1) the usage of "political mobilization, recurrent rhetoric and symbols designed to inspire the people"; 2) the formation of a heterogeneous coalition that while includes and is led by the middle class and even part of elites, is aimed to advance the interests of the working class; this is done by (3) a reformist agenda – usually expanding the role of the state to accelerate industrialization – to promote development allied with redistributive policies.

Since then, we may highlight the work of de la Torre \parencite*{DelaTorre2016,DelaTorre2017}, Weyland \parencite*{Weyland2001,Weyland2009}, Doyle \parencite*{Doyle2011}  and that of Edwards \parencite*{Edwards2010,Edwards2019} himself\footnote{See also Kaltwasser \textit{et al.} \parencite*{Kaltawasser2017} for a comprehensive review of the subject.}. We also should mention works that try to \textit{measure} populism such as Hawkins \parencite*{Hawkins2009}, Hawkins, Aguilar, and Castanho \parencite*{Hawkins2019}, and Ramos \parencite*{MRamos2018}. 

Populists usually are charismatic individuals, who lead a personalistic government. They rely on a "us versus them" rhetoric, where the populist leader "saves" the people – conceived as an unique and homogeneous body\footnote{Though, see de la Torre \parencite*{DelaTorre2016} for examples of "the People" conceived as heterogeneous.} – from the abuses of a ruling elite. In opposition to the rule of law, "The People" is also the ultimate source of legitimacy \parencite{Abts2007}. This narrative also has a nationalistic component \parencite{DelaTorre2017a}, given that the elites are generally comprised by large multinational corporations, and organisations such as the International Monetary Fund, notoriously.

To use Dornbusch \& Edwards \parencite*[][p. 247] own definition: "Macroeconomic populism is an approach to economics that emphasizes growth and income distribution and deemphasizes the risks of inflation and deficit finance, external constraints and the reaction of economic agents to aggressive non-market policies". Put simply, populism is a violation of the "good economics" of fiscal responsibility, budget constraints, and efficiency \parencite{Edwards2019}. 

Hence, populist macroeconomics starts with a common diagnosis of the country's economic issues. These are the three \textit{R}'s of populist macroeconomics: \textit{reactivate, redistribute, restructure}. Although not necessary, the initial conditions comprises an unequal society claiming for political change, since stagnation or insufficient growth is maintaining standards of living at undesirable levels. Previous stabilization attempts, often guided by a program elaborated by the IMF, may have given some leeway for expansionist policies. 

Crucially, then, is the understanding that the country has a hiatus on production, i.e. there is spare capacity in the industry that should allow for aggressive expansionist policies without the dangers of inflation. If the disease is an idle industry, the solution is to first, \textit{reactivate} it. Aligned with the expansionist policies is the \textit{redistribution} of income, through significant increases in real wages\footnote{Though Dornbusch \& Edwards \parencite*{Dornbusch1990} do not point out to this, the redistribution of income has two purposes. The economic rationale is that lower income brackets employ proportionally more of their income into spending, while the elites are only "extracting rents" – \textit{los rentistas} ("the rentists") are culprits that can be found in every crime scene of Latin American politics. The second related purpose is to appeal for the political narrative of "us versus them", of inequality and exploitation of the people by the elites.}. Finally, \textit{restructure} the domestic economy, especially by the industrialization of the economy\footnote{However, de la Torre \parencite*[][p. 196]{DelaTorre2017} argues that industrialization was mainly intended for relatively more developed nations as Brazil, Argentina, and Mexico, not being found in populist policies in Bolivia, Ecuador and Peru.}. The way to achieve industrialization, as promoted by the United Nations Economic Commission for Latin America and the Caribbean ("Económica para América Latina y el Caribe", CEPAL), was through subsidies to the industry and import substitution industrialization policies\footnote{These ideas were famously advocated by scholars such as Raúl Prebisch and Celso Furtado. For a discussion of these ideas, see Aguilar \parencite*{Aguilar1986}. See Prebisch (e.g. 1976;1987). Also see Ocampo \parencite*{JAOCampo2001} for a complete review of Prebisch's ideas. For Furtado, see his \textit{Desenvolvimento e Subdesenvolvimento} \parencite*{Furtado1961}).}.

Following, Dornbusch \& Edwards argue that the implementation of populist policies generally ends up having four stages, or steps. 

\textit{1.} In the beginning, the macroeconomic scenario responds successfully. Expansion leads to growing output, real wages and employment also follow. Since there are inventories to be drawn and reserves to be used to finance imports, there is no danger of shortages and inflation. 

\textit{2.} Not long afterwards (around a year later), as foreign reserves start to reach critical levels, imports  are reduced, affecting both final and capital goods, the latter also impacting the domestic industry with lacking inputs for production. Bottlenecks start to appear everywhere; shortages become common and inflation rises fast. Exchange rate manipulation, subsidies and price controls are attempted to stabilize the economy. 

\textit{3.} As wages are being constantly increase to keep up with inflation, massive subsidies become necessary to boost domestic production, deteriorating the fiscal situation of the country, with increasing budget deficits. Capital flights away and the economy collapses. Chaotic shortages result, accompanied by severe deficits, since tax collection declines with decreasing economic activity. When the economy finally collapses, real wages fall rapidly, as subsidies are cut and the economy shuts down.

\textit{4.} A new government assumes and implements orthodox policies for stabilization, probably an IMF program. With a strong retraction, real wages sharply fall, and the country is at an income level lower than before. Since capital has flown overseas, the impact is enduring, and the economy struggles to rebuild. The comeback depends on local politics allowing for the adjustment process; the credibility of the new government will dictate if foreign investments will be attracted to rebuild.

% ================================================================
% --- SECTION 3: POPULISM IN THE 21ST CENTURY
\section{Populism in the 21st Century} 
    \label{sec:section_3}
    
We investigate if and exactly how this thesis applies to 21st populism in Latin America. Even though there is no unique definition of populism, there is a fair consensus of what countries that had populism governments\footnote{See (e.g.) Absher, Grier, \& Grier \parencite*{Absher2020}, Colburn \& Cruz \parencite*{Colburn2012} \parencite*{delaTorre2013,delaTorre2017a}, Ocampo (year), Bittencourt \parencite*{Bittencourt2012}, Doyle \parencite*{Doyle2011}. While Brazil under the Worker's Party of Lula (2003-10) and Dilma (2011-16) are potential candidates, especially regarding their \textit{policies}, they are significantly less mentioned in the literature}.

\textbf{Talk about "Neo-Populism"?}
Based on the literature, we select 5 populist governments\footnote{For another discussion of populism in these countries, see Cachanosky \& Padilla \parencite*{Cachanosky2019c}.}. Argentina under the Kirchner's, that is, Néstor Kirchner (2003-07) and his wife Cristina Kirchner (2007-15); the government of Evo Morales (2006-2019) in Bolivia; Ecuador under the presidency of Rafael Correa (2007-2016); the Nicaragua of Daniel Ortega (2007-present) and the Bolivarian Venezuela of Hugo Chávez (1999-2013) and Nicolás Maduro (2013-2019)\footnote{On January 10, 2019, the National Assembly declared the 2018 presidential elections as invalid due to fraud, and its president, Juan Guaidó assumed as interim president. The presidency has been disputed so far.}. 

\subsection{Argentina}

\subsubsection{The economic policy}

Néstor Kirchner was elected as president in May 2003, less than two years after the largest economic crisis in the history of Argentina: the 2001 Argentine Great Depression. Both, Néstor and his wife, Cristina Fernández Kirchner (CFK), were elected through the political party Partido Justicialista (PJ), founded in 1947 by Juan Perón, one of the fathers of Latin America "golden age" of populism \parencite{DelaTorre2017, Gambini1999}\footnote{See also Ocampo \parencite*{Ocampo2015,Ocampo2015a}}. The Peronists were in charge of the presidency for 37 out 75 years since then; they received between 30\% and over 60\% of votes in presidential and parliamentary elections, while also governing multiple provinces and municipalities around the country. It is not a far stretch to describe Argentina as a peronist country.

Five presidents, four of them from the PJ, were in office between de la Rúa's (UCR -Unión Cívica Radical) resignation on December 20th, 2001, and Néstor's presidential term.\footnote{After de la Rúa, Ramón Puerta (PJ) assumes for a period of two days, being replaced by Adolfo Rodriguez Saá (PJ) on December 22nd, who declares what was the largest default in history at the time. A week later, on December 30th, he is forced out of office. Finally, in January 2nd, Congress appoints Eduardo Duhalde (PJ) as the President to complete de la Rua's original term. When new elections were called in 2003, Néstor wins when Menem drops from running in the ballotage}. After this political crisis, Néstor assumes with ideal conditions to start the typical \textit{stage 1} of macroeconomic populism. The 2001 crisis left behind (1) significant output gaps, (2) high inflation, (3) high poverty rates, and (4) a high social demand for political leaders to \textit{do something}. Néstor also received a rarity in Argentina, a budget surplus due to the 2001 default and tax hikes. Néstor and Cristina were able to build political power from the institutional crisis and delivered on the public demands.\footnote{For a more detailed treatment of the discussion that follows see Sturzenegger \parencite*{Sturzenegger2019}, and Thomas and Cachanosky \parencite*{Thomas2015}.} The output gaps left behind by the crisis allowed for an easy recovery of the economy, aided by a boom in the price of commodities.

In terms of measures to fight a spike in poverty rates, two type of policies implemented after the 2001 crisis were continued by the kirchner administration. The first one was a new set of social programs targeting low income households and unemployed. The second one was freezing the tariff of utilities (such as gas, energy, and public transportation) at the pre-crisis levels. Namely, utility providers would sell domestically in pesos at pre-crisis prices but would have to import their inputs paying the new depreciated exchange rate. These two policies became permanent and are still present today, two decades after the 2001 crisis. 

Social programs and subsidies became a high burden to the treasury, which went into deficit in 2007 to never see a surplus again. Between 2003 and 2015, these two items represented 65\% of total government spending. With international markets closed to Argentina, the kirchners turned to different sources to finance the deficit such as confiscations and monetization. The last one brought a new cycle of high inflation to Argentina starting the \textit{stage 2} of macroeconomic populism. A 9\% inflation rate in 2007 grew steadily to 50\% in 2020. Néstor's presidency had an average inflation rate of 15\%. Cristina's two-term presidency had an average inflation rate of 25.6\%. The kirchners used a left-populist rhetoric to build a strong support group that would welcome the confiscation of big corporations and blame inflation on big businesses or international factors.\footnote{In 2008, CFK nationalize the private retirement and pension funds, and in 2012, she nationalized the Spanish participation in the oil company Repsol-YPF.}

Utility prices (energy, gas, transportation) ran behind inflation because they were frozen during the 2001 crisis. These price controls brought (1) higher consumption of energy and (2) lower production of energy. The shortage was satisfied with imports and some unplanned rationing such as energy outages. The import of energy and other services became a drain to the central bank reserves. In terms of total merchandise traded, Argentina exported 17.1\% and imported 3.5\% in 2003 (a positive difference of 13.6\%). In 2004, exports where 4.7\% and imports 16.6\% (a negative difference of 11.9\%). We identify the beginning of \textit{stage 3} of macroeconomic populism with the imposition of strict capital controls. The magnitud of reserve loses is significant. During the 2001 crisis, the Argentine central bank lost 3,679 millions USD in international reserves. During the capital controls between 2011 and 2015, the central bank lost 22,678 million USD in international reserves (six times the amount lost in 2001). The reserves gain from the import of commodities during a record of high international prices was not enough to pay for the kirchners' populist policies.

The Kirchner's regime ends in 2015, when Macri becomes President with the political coalition Cambiemos. By the time Macri takes office, the inflation rate was 27.8\% (with an upward trend) and the exchange rate was over-appreciated. The central government deficit was a 6.9\% of GDP (two times as large as it was in 2001). The real economy in stagfaltion since 2011. The Pontificia Universidad Católica Argentina estimated a household poverty rate of 30\%.\footnote{Becuase of the kirchners tampering with official statistics, there are no reliable official poverty rate estimations for this period.}

The beginning of Macri's president marks the beginning of macroeconomic populism \textit{stage 4}. In real terms, government spending decreased 20\% under Macri's 4-year presidency. The first year of Macri's presidency the Treasury's deficit raised to 9.5\% of GDP, but his last year in office ended with a deficit of 4.7\% of GDP. The economy remained stagnated and the inflation rate continued its upward trend, with an average rate of 41\% for Macri's presidential term.\footnote{During Macri's presidency the new authorities of the central bank implemented an inflation targeting regime that broke down within two years. See Cachanosky \parencite*{Cachanosky2021a} and Sturzenegger \parencite*{Sturzenegger2017, Sturzenegger2020}.} Macri's government also relied on the International Monetary Fund to secure financial help during the currency crises of 2018.

The end of Macri's presidency looks like a beginning of a new round of the 4 stages of macroeconomic populism. Macri's loses the presidential elections for a second term against Alberto Fernández, former Chief of the Cabinet of Ministers under the Kirchner's between May 2003, and August 2008. Once again an economic crisis, in this cases in the form of a currency crisis in 2018, precedes the election of a populist leader.

\begin{table}[!h]
\begin{center}
\caption{Argentina. Stages of macroeconomic populism} \label{table:ARG_Stages}
\begin{tabular}{l l l l l l} \\ \toprule
  Stages   & Start     & End       & Length             & End of stage event     & President(s)  \\ \midrule
  Stage 1  & May, 2003 & Jan, 2007 & 3 years, 8 months  & Inflation starts       & Néstor \& CFK \\
  Stage 2  & Jan, 2007 & Oct, 2011 & 4 years, 9 months  & Capital controls       & CFK           \\
  Stage 3  & Oct, 2011 & Jan, 2016 & 4 years, 3 months  & Presidential elections & CFK           \\
  Stage 4  & Jan, 2016 & Jan, 2020 & 4 years            & Return of populism     & Macri         \\ \midrule
  Total    &           &           & 12 years, 8 months &                        &               \\
  \bottomrule 
\end{tabular}
\end{center}
\end{table}

\subsubsection{The four stages of macroeconomic populism}

We propose the following 4 stages of macroeconomic populism for the argentine case. \textit{Stage 1} begins with Néstor´s presidency in 2003 and ends with the start of inflation in 2007. \textit{Stage 2} ends with the imposition of capital controls in 2011. \textit{Stage 3} ends with the beginning of Macri's presidency (2016-2020). Table \ref{table:ARG_Stages} shows the four stages of macroeconomic populism in Argentina and the following tables show the behavior of key economic variables.

%\begin{table}[!h]
\begin{center}
\caption{Argentina. Stages of macroeconomic populism} \label{table:ARG_Stages}
\begin{tabular}{l l l l l l} \\ \toprule
  Stages   & Start     & End       & Length             & End of stage event     & President(s)  \\ \midrule
  Stage 1  & May, 2003 & Jan, 2007 & 3 years, 8 months  & Inflation starts       & Néstor \& CFK \\
  Stage 2  & Jan, 2007 & Oct, 2011 & 4 years, 9 months  & Capital controls       & CFK           \\
  Stage 3  & Oct, 2011 & Jan, 2016 & 4 years, 3 months  & Presidential elections & CFK           \\
  Stage 4  & Jan, 2016 & Jan, 2020 & 4 years            & Return of populism     & Macri         \\ \midrule
  Total    &           &           & 12 years, 8 months &                        &               \\
  \bottomrule 
\end{tabular}
\end{center}
\end{table}

In terms of real GDP (figure \ref{fig:ARG_GDP}), stage 1 benefits from a bounce-back from the 2001 crisis and the rise in the price of commodities \parencite{Ocampo2015, Thomas2015}. Growth continues in stage 2 but stagnates at the beginning of stage 3 in 2011. Inflation starts in 2007, therefore starting in 2011 Argentina is in stagflation, situation that remains through all of stage 4 (2016 - 2020) as well.

\begin{figure}[!h]
    \caption{Real GDP (2017 prices)}
    \centering
    %\includegraphics[width=\textwidth]{Images/ARG_GDP.png}
    \footnotesize{\textit{Source}: Penn World Table 10.0.}
    \label{fig:ARG_GDP}
\end{figure}

Real wages (figure \ref{fig:ARG_RealWage}) follow a similar pattern to real GDP with a clearer decline starting in 2017 (mid of stage 4). Real wage behavior follows Dornbusch \& Edwards \parencite*{Dornbusch1990} narrative, with the difference that they state that real wages fall to a level lower than the beginning of the populist cycle. A reason real wages do not fall to pre-populist levels is because of the impact of high commodity prices in GDP in the real wage estimation.\footnote{We estimate real wages using Penn World Table 10.0 data in the following way: Real wages $(w/P)$ equal the share of labor compensation in GDP at current national prices$(\omega)$ times real GDP at 2017 national prices $(Y)$ divided by the number of persons engaged $(N)$: $\frac{w}{P} = \frac{\omega Y}{N}$.}

\begin{figure}[!h]
    \caption{Real wages}
    \centering
    %\includegraphics[width=\textwidth]{Images/ARG_RealWage.png}
    \footnotesize{\textit{Source}: Authors' estimation based on Penn World Table 10.0.}
    \label{fig:ARG_RealWage}
\end{figure}

Government spending (figure \ref{fig:ARG_GovSpending}) also maps the four stages of macroeconomic populism, going from 13\% of GDP (2003) to 24\% of GDP at the end of the kirchner administration (2015). The growth of government's size is steady and continues even after the economy falls into stagflation. Austerity measures, in terms of government spending over GDP, start in the second year of Macri's administration. After increasing the size of the state to 26\% in the first year, it then goes down to 22\% of GDP by 2019. The first year increase and the slow adjustment that follows is the result of the gradualist approach to shrinking the size of the government of Macri's government. Macri's gradualist approach was at odds with the central bank intention of a quick reduction of the inflation through an inflation targeting regime. The inconsistencies in the economic policy finally led to a currency crisis and return of the kirchneristas to the government in 2020, with Alberto Fernández (former Chief of Cabinet of Ministers during Néstor's presidency) as President and CFK as Vice-President \parencite{Cachanosky2021a, Sturzenegger2019}. 

\begin{figure}[!h]
    \caption{Government Spending (\%GDP)}
    \centering
    %\includegraphics[width=\textwidth]{Images/ARG_GovSpending.png}
    \footnotesize{\textit{Source}: European Commission for Latin America and the Caribbean (ECLAC)}
    \label{fig:ARG_GovSpending}
\end{figure}

Central bank reserves (figure \ref{fig:ARG_Reserves}) increase during stage 1 of macroeconomic populism, but stagnate through all of stage 2. Despite strict capital controls, central bank reserves start fall all of stage 3 only to increase again in stage 4. The sharp fall of reserves in 2018 is due to the currency crisis and the breakdown of the inflation targeting regime. The behavior of the central bank reserves offer a close match to the stylized 4 stages of macroeconomic populism.

\begin{figure}[!h]
    \caption{Central bank reserves}
    \centering
    %\includegraphics[width=\textwidth]{Images/ARG_Reserves.png}
    \footnotesize{\textit{Source}: ???}
    \label{fig:ARG_Reserves}
\end{figure}

In terms of inflation (figure \ref{fig:ARG_Inflation}), the series shows an upward trend that surpasses the 40\% peak of the 2001 crisis. The high inflation cycle starts in 2007, when inflation jumps from 9.8\% to 25.7\%. Nestor's average inflation rate was 15\%, CFK's was 25.6\%, and Macri's was 41\%. The year 2007 is also when the kirchner administration starts to tamper with official inflation numbers, interference that did not stop until Macri's presidency starts in 2016.\footnote{The tampered inflation numbers are replaced with a composite of private estimations that where published as "Inflación Congreso."} The initial attempts to reduce inflation through an inflation targeting regime seems to have results, as inflation falls from 40.7\% in 2016 to 24.7\% in 2017. However, the situation soon runs out of control and the inflation rate surpasses the highest levels of the kirchner administration.

\begin{figure}[!h]
    \caption{Inflation}
    \centering
    %\includegraphics[width=\textwidth]{Images/ARG_Inflation.png}
    \footnotesize{\textit{Source}: Inflación Congreso (manually collected by the authors) and Instituto de Estadísticas y Censos (INDEC)}
    \label{fig:ARG_Inflation}
\end{figure}

As explained in the previous section, the energy sector was particularly affected by the kirchner populist policies. Argentina went from being a net export of energy to a net import by the end of stage 2 (figure \ref{fig:ARG_Energy}. This energy trade deficit became an important drain for the central bank reserves. 

\begin{figure}[!h]
    \caption{Next exports of energy (\% of merchandise exports)}
    \centering
    %\includegraphics[width=\textwidth]{Images/ARG_Energy.png}
    \footnotesize{\textit{Source}: World Bank, World Development Indicators}
    \label{fig:ARG_Energy}
\end{figure}


\subsection{Bolivia}

Evo Morales was the Bolivian president for three consecutive from 2006 to 2019, when he resigned. Bolivia's economic performance during his initial mandate was praised by some authors as "remarkable", in part due to a "large-scale and well-timed increase in public spending" \parencite[see][p. 6]{Weisbrot2009} (see Weisbrot, Ray, \& Johnston, 2009, p. 6). 


\subsection{Ecuador}


\subsection{Nicaragua}


\subsection{Venezuela}

Venezuela is indisputably the flagship example of Latin American populism. Before getting to power, Chávez attempted a coup d'état against Carlos Andrés Pérez in 1992, and was arrest after its failure. After being pardoned two years later, he founded the Fifth Republic Movement (Movimiento V [Quinta] República, MVR), and ran for president, winning the 1998 elections. 

Chávez assumed power in 1999 and immediately sought to promote referendum to call a constitutional assembly. At the constitutional assembly opposition held merely 6 out 125 seats \parencite[p.130]{Marcano2007}, and the whole process, from the draft to its date effective, lasted 33 days. The "Bolivarian Constitution" promote significant changes: it changed the name of the country to \textit{Bolivarian} Republic of Venezuela, transformed the bi-cameral legislative into a unicameral one system, and greatly expanded the powers of the Executive. More importantly, it prolonged the presidential term from 5 to 6 years and allowed re-elections.

In 2002 elections, Chávez was re-elected and his supporters won 101 out of 165 seats for the National Assembly. This majority shortly after gave Chávez the capacity to rule by decree. In the same year Chávez suffered 

In 2006, the Law of Communal Councils (\textit{consejos comunales}) was approved, creating mechanisms of participatory democracy. Councils could opine and oversee local policies. Many communal councils became \textit{colectivos}, armed militias that defend the interests of the government. (https://es.insightcrime.org/wp-content/uploads/2018/05/Venezuela-a-Mafia-State-InSight-Crime-2018.pdf)


From wiki:
Over 19,500[2] councils were registered throughout the country and billions of dollars have been distributed to support their efforts by the government.[3]



Populist policies reduce economic freedom, namely through increased regulation, reduced trade, and less secured property rights. 

% ================================================================
% --- SECTION 4: DISCUSSION
% \section{Discussion} 
%    \label{sec:section_4}

% ================================================================
% --- SECTION 5: REFERENCES
\newpage

\singlespacing
\printbibliography

% ================================================================
% --- THE END
\end{document}