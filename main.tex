% ================================================================
% ================================================================
% --- PREAMBLE
% --- Document class
\documentclass[12pt, letterpaper]{article}

% --- Packages
\usepackage[utf8]{inputenc}
\usepackage[top=1in, bottom=1in, left=1in, right=1in]{geometry}
\usepackage{setspace}
\usepackage{microtype}
\usepackage[dvipsnames]{xcolor}
\usepackage{lastpage}
\usepackage{hyperref}
\usepackage{fancyhdr}
\usepackage{booktabs}
\usepackage{graphicx}
\usepackage{todonotes}
\usepackage{fontspec}
\usepackage{adjustbox}
\usepackage{float}
\usepackage{dcolumn}
\usepackage[backend=biber, style=apa, citestyle=apa]{biblatex}

% --- Settings
\usepackage{libertine}
\usepackage[libertine]{newtxmath}
\addbibresource{references.bib}

\hypersetup{
    colorlinks = true,
    linkcolor  = blue,
    urlcolor   = blue,
    citecolor  = blue,
    }
    
\pagestyle{fancy}

% --- Header and footer
\fancyhf{}
\headheight = 28pt
\chead{\small{Macroeconomic Populism in the 21st Century: Revisiting Dornbusch \& Edwards\\
       \textit{N. Cachanosky, J. P. Bastos, T. Faintich}}}
\cfoot{\footnotesize Page \thepage \hspace{0.5pt} of \pageref{LastPage}} 

% --- Title page
\title{Macroeconomic Populism in the 21st Century: Revisiting Dornbusch \& Edwards
       \bigskip}

\author{
        \textbf{Nicolás Cachanosky} \\
        Metropolitan State University of Denver \\
        Department of Economics \\
        \href{mailto:ncachano@msudenver.edu}{ncachano@msudenver.edu} \\
        \and
        \textbf{João Pedro Bastos} \\
        Texas Tech University \\
        Free Market Institute \\
        \href{mailto:joao-pedro.bastos@ttu.edu}{joao-pedro.bastos@ttu.edu} \\
        \and
        \textbf{Tomás Faintich} \\
        Universidad del CEMA \\
        Friedman-Hayek Center for the Study of a Free Society \\
        \href{mailto:tfaintich22@ucema.edu.ar}{tfaintich22@ucema.edu.ar}
        \bigskip}

\date{\today}


% ================================================================
% ================================================================
% --- DOCUMENT
\setlength{\marginparwidth}{2cm}
\begin{document}


% ================================================================
% --- TITLE PAGE

\maketitle

\begin{abstract}
\noindent
The seminal work of Dornbusch \& Edwards \parencite*{Dornbusch1990} described that the mechanism through which populist leaders stepped into power and implemented their policies in Latin American had common features along different countries. Their work discusses the similarities of populist governments in Peru and Chile. We replicate their findings for the 21st century Latin America, discussing the cases of Argentina, Bolivia, Ecuador, Nicaragua, and Venezuela.
\end{abstract}

\bigskip \bigskip
\footnotesize \noindent \textbf{JEL codes}: TBD \\
\footnotesize \noindent \textbf{Keywords}: Latin America, Populism, Inflation.

\newpage
\doublespacing

% ================================================================
% --- SECTION 1: INTRODUCTION
\section{Introduction} 
    \label{sec:Intro}

The seminal work of Dornbusch \& Edwards \parencite*{Dornbusch1990} advanced the idea that while different in many aspects – ideology included – populism governments have a common economic agenda. As they called, "macroeconomic populism" is defined by governments that pursued growth and income distribution at all costs, neglecting the negative effects of deficits and inflation. 

Their work provided two case studies, that of Chile under the Unidad Popular (UP) of Salvador Allende (1970-73), and Peru under Alán Garcia (1985-1990). Indeed, while notoriously different these two governments shared "critical economic factors". Starting amid under-performing economic growth, both proposed a "reactivation" of the economy, redistribution of income, and restructuring the industry. These happened in a similar fashion in both countries.  

Since the economy was under-performing, spare capacity in the industry should, in theory, allow for a "safe" monetary expansion, without inflationary pressure. Real wages went up, promoting redistribution of income. The economy, at first, took off. But such growth is unsustainable. The demand shock is not accompanied by a real increase in production. Imports fall as international reserves diminish, impacting both supply of final goods and of inputs for the domestic industry. Inflation rises fast and shortages occur everywhere. The situation quickly deteriorates, sharply reducing income, and chaos is installed.

Looking at Argentina under the Kirchners, Bolivia under Evo Morales, the Ecuador of Rafael Correa, Daniel Ortega's Nicaragua, and the "Bolivarian" Venezuela of Chavez and Maduro, we analyze the macroeconomic populism thesis for the twenty-first Latin America. 

Section 2 discusses Dornbusch \& Edwards's theory in detail. Section 3 discusses our selection of countries and the policies they implemented. Section N concludes. 


% ================================================================
% --- SECTION 2: MACROECONOMIC POPULISM THESIS
\section{The "Macroeconomic Populism" Thesis} 
    \label{sec:MacroPopulism}
    
Political science has been struggling to provide a consistent definition of populism. Rather then a specific type of government with a strict set of policies, the field has provided us with a bundle of policies and characteristics that most populist governments have. At the time, Dornbusch \& Edwards \parencite*{Dornbusch1990} relied mostly in the work of Drake \parencite*{Drake1982} which proposes that populist are characterized by 1) the usage of "political mobilization, recurrent rhetoric and symbols designed to inspire the people"; 2) the formation of a heterogeneous coalition that while includes and is led by the middle class and even part of elites, is aimed to advance the interests of the working class; this is done by (3) a reformist agenda – usually expanding the role of the state to accelerate industrialization – to promote development allied with redistributive policies.

Since then, we may highlight the work of de la Torre \parencite*{DelaTorre2016,DelaTorre2017}, Weyland \parencite*{Weyland2001,Weyland2009}, Doyle \parencite*{Doyle2011}  and that of Edwards \parencite*{Edwards2010,Edwards2019} himself\footnote{See also Kaltwasser \textit{et al.} \parencite*{Kaltawasser2017} for a comprehensive review of the subject.}. We also should mention works that try to \textit{measure} populism such as Hawkins \parencite*{Hawkins2009}, Hawkins, Aguilar, and Castanho \parencite*{Hawkins2019}, and Ramos \parencite*{MRamos2018}. 

Populists usually are charismatic individuals, who lead a personalistic government. They rely on a "us versus them" rhetoric, where the populist leader "saves" the people – conceived as an unique and homogeneous body\footnote{Though, see de la Torre \parencite*{DelaTorre2016} for examples of "the People" conceived as heterogeneous.} – from the abuses of a ruling elite. In opposition to the rule of law, "The People" is also the ultimate source of legitimacy \parencite{Abts2007}. This narrative also has a nationalistic component \parencite{DelaTorre2017a}, given that the elites are generally comprised by large multinational corporations, and organisations such as the International Monetary Fund, notoriously.

To use Dornbusch \& Edwards \parencite*[][p. 247] own definition: "Macroeconomic populism is an approach to economics that emphasizes growth and income distribution and deemphasizes the risks of inflation and deficit finance, external constraints and the reaction of economic agents to aggressive non-market policies". Put simply, populism is a violation of the "good economics" of fiscal responsibility, budget constraints, and efficiency \parencite{Edwards2019}. 

Hence, populist macroeconomics starts with a common diagnosis of the country's economic issues. These are the three \textit{R}'s of populist macroeconomics: \textit{reactivate, redistribute, restructure}. Although not necessary, the initial conditions comprises an unequal society claiming for political change, since stagnation or insufficient growth is maintaining standards of living at undesirable levels. Previous stabilization attempts, often guided by a program elaborated by the IMF, may have given some leeway for expansionist policies. 

Crucially, then, is the understanding that the country has a hiatus on production, i.e. there is spare capacity in the industry that should allow for aggressive expansionist policies without the dangers of inflation. If the disease is an idle industry, the solution is to first, \textit{reactivate} it. Aligned with the expansionist policies is the \textit{redistribution} of income, through significant increases in real wages\footnote{Though Dornbusch \& Edwards \parencite*{Dornbusch1990} do not point out to this, the redistribution of income has two purposes. The economic rationale is that lower income brackets employ proportionally more of their income into spending, while the elites are only "extracting rents" – \textit{los rentistas} ("the rentists") are culprits that can be found in every crime scene of Latin American politics. The second related purpose is to appeal for the political narrative of "us versus them", of inequality and exploitation of the people by the elites.}. Finally, \textit{restructure} the domestic economy, especially by the industrialization of the economy\footnote{However, de la Torre \parencite*[][p. 196]{DelaTorre2017} argues that industrialization was mainly intended for relatively more developed nations as Brazil, Argentina, and Mexico, not being found in populist policies in Bolivia, Ecuador and Peru.}. The way to achieve industrialization, as promoted by the United Nations Economic Commission for Latin America and the Caribbean ("Económica para América Latina y el Caribe", CEPAL), was through subsidies to the industry and import substitution industrialization policies\footnote{These ideas were famously advocated by scholars such as Raúl Prebisch and Celso Furtado. For a discussion of these ideas, see Aguilar \parencite*{Aguilar1986}. See Prebisch (e.g. 1976;1987). Also see Ocampo \parencite*{JAOCampo2001} for a complete review of Prebisch's ideas. For Furtado, see his \textit{Desenvolvimento e Subdesenvolvimento} \parencite*{Furtado1961}).}.

Following, Dornbusch \& Edwards argue that the implementation of populist policies generally ends up having four stages, or steps. 

\textit{1.} In the beginning, the macroeconomic scenario responds successfully. Expansion leads to growing output, real wages and employment also follow. Since there are inventories to be drawn and reserves to be used to finance imports, there is no danger of shortages and inflation. 

\textit{2.} Not long afterwards (around a year later), as foreign reserves start to reach critical levels, imports  are reduced, affecting both final and capital goods, the latter also impacting the domestic industry with lacking inputs for production. Bottlenecks start to appear everywhere; shortages become common and inflation rises fast. Exchange rate manipulation, subsidies and price controls are attempted to stabilize the economy. 

\textit{3.} As wages are being constantly increase to keep up with inflation, massive subsidies become necessary to boost domestic production, deteriorating the fiscal situation of the country, with increasing budget deficits. Capital flights away and the economy collapses. Chaotic shortages result, accompanied by severe deficits, since tax collection declines with decreasing economic activity. When the economy finally collapses, real wages fall rapidly, as subsidies are cut and the economy shuts down.

\textit{4.} A new government assumes and implements orthodox policies for stabilization, probably an IMF program. With a strong retraction, real wages sharply fall, and the country is at an income level lower than before. Since capital has flown overseas, the impact is enduring, and the economy struggles to rebuild. The comeback depends on local politics allowing for the adjustment process; the credibility of the new government will dictate if foreign investments will be attracted to rebuild.

% ================================================================
% --- SECTION 3: POPULISM IN THE 21ST CENTURY
\section{Populism in the 21st Century} 
    \label{sec:Countries}
    
We investigate if and exactly how this thesis applies to 21st populism in Latin America. Even though there is no unique definition of populism, there is a fair consensus of what countries that had populism governments\footnote{See (e.g.) Absher, Grier, \& Grier \parencite*{Absher2020}, Colburn \& Cruz \parencite*{Colburn2012} \parencite*{delaTorre2013,delaTorre2017a}, Ocampo (year), Bittencourt \parencite*{Bittencourt2012}, Doyle \parencite*{Doyle2011}. While Brazil under the Worker's Party of Lula (2003-10) and Dilma (2011-16) are potential candidates, especially regarding their \textit{policies}, they are significantly less mentioned in the literature}.

\textbf{Talk about "Neo-Populism"?}
Based on the literature, we select 5 populist governments\footnote{For another discussion of populism in these countries, see Cachanosky \& Padilla \parencite*{Cachanosky2019c}.}. Argentina under the Kirchner's, that is, Néstor Kirchner (2003-07) and his wife Cristina Kirchner (2007-15); the government of Evo Morales (2006-2019) in Bolivia; Ecuador under the presidency of Rafael Correa (2007-2016); the Nicaragua of Daniel Ortega (2007-present) and the Bolivarian Venezuela of Hugo Chávez (1999-2013) and Nicolás Maduro (2013-2019)\footnote{On January 10, 2019, the National Assembly declared the 2018 presidential elections as invalid due to fraud, and its president, Juan Guaidó assumed as interim president. The presidency has been disputed so far.}. 

%==============================================
\subsection{Argentina}

\subsubsection{The economic populist-policies}

Néstor Kirchner was elected as president in May 2003, less than two years after the largest economic crisis in the history of Argentina: the 2001 Argentine Great Depression. Both, Néstor and his wife, Cristina Fernández Kirchner (CFK), were elected through the political party Partido Justicialista (PJ), founded in 1947 by Juan Perón, one of the fathers of Latin America "golden age" of populism \parencite{DelaTorre2017, Gambini1999}\footnote{See also Ocampo \parencite*{Ocampo2015,Ocampo2015a}}. The Peronists were in charge of the presidency for 37 out 75 years since then; they received between 30\% and over 60\% of votes in presidential and parliamentary elections, while also governing multiple provinces and municipalities around the country. It is not a far stretch to describe Argentina as a peronist country.

Five presidents, four of them from the PJ, were in office between de la Rúa's (UCR -Unión Cívica Radical) resignation on December 20th, 2001, and Néstor's presidential term.\footnote{After de la Rúa, Ramón Puerta (PJ) assumes for a period of two days, being replaced by Adolfo Rodriguez Saá (PJ) on December 22nd, who declares what was the largest default in history at the time. A week later, on December 30th, he is forced out of office. Finally, in January 2nd, Congress appoints Eduardo Duhalde (PJ) as the President to complete de la Rua's original term. When new elections were called in 2003, Néstor wins when Menem drops from running in the ballotage}. After this political crisis, Néstor assumes with ideal conditions to start the typical \textit{stage 1} of macroeconomic populism. The 2001 crisis left behind (1) significant output gaps, (2) high inflation, (3) high poverty rates, and (4) a high social demand for political leaders to \textit{do something}. Néstor also received a rarity in Argentina, a budget surplus due to the 2001 default and tax hikes. Néstor and Cristina were able to build political power from the institutional crisis and delivered on the public demands.\footnote{For a more detailed treatment of the discussion that follows see Sturzenegger \parencite*{Sturzenegger2019}, and Thomas and Cachanosky \parencite*{Thomas2015}.} The output gaps left behind by the crisis allowed for an easy recovery of the economy, aided by a boom in the price of commodities.

In terms of measures to fight a spike in poverty rates, two type of policies implemented after the 2001 crisis were continued by the kirchner administration. The first one was a new set of social programs targeting low income households and unemployed. The second one was freezing the tariff of utilities (such as gas, energy, and public transportation) at the pre-crisis levels. Namely, utility providers would sell domestically in pesos at pre-crisis prices but would have to import their inputs paying the new depreciated exchange rate. These two policies became permanent and are still present today, two decades after the 2001 crisis. 

Social programs and subsidies became a high burden to the treasury, which went into deficit in 2007 to never see a surplus again. Between 2003 and 2015, these two items represented 65\% of total government spending. With international markets closed to Argentina, the kirchners turned to different sources to finance the deficit such as confiscations and monetization. The last one brought a new cycle of high inflation to Argentina starting the \textit{stage 2} of macroeconomic populism. A 9\% inflation rate in 2007 grew steadily to 50\% in 2020. Néstor's presidency had an average inflation rate of 15\%. Cristina's two-term presidency had an average inflation rate of 25.6\%. The kirchners used a left-populist rhetoric to build a strong support group that would welcome the confiscation of big corporations and blame inflation on big businesses or international factors.\footnote{In 2008, CFK nationalize the private retirement and pension funds, and in 2012, she nationalized the Spanish participation in the oil company Repsol-YPF.}

Utility prices (energy, gas, transportation) ran behind inflation because they were frozen during the 2001 crisis. These price controls brought (1) higher consumption of energy and (2) lower production of energy. The shortage was satisfied with imports and some unplanned rationing such as energy outages. The import of energy and other services became a drain to the central bank reserves. In terms of total merchandise traded, Argentina exported 17.1\% and imported 3.5\% in 2003 (a positive difference of 13.6\%). In 2004, exports where 4.7\% and imports 16.6\% (a negative difference of 11.9\%). We identify the beginning of \textit{stage 3} of macroeconomic populism with the imposition of strict capital controls. The magnitud of reserve loses is significant. During the 2001 crisis, the Argentine central bank lost 3,679 millions USD in international reserves. During the capital controls between 2011 and 2015, the central bank lost 22,678 million USD in international reserves (six times the amount lost in 2001). The reserves gain from the import of commodities during a record of high international prices was not enough to pay for the kirchners' populist policies.

The Kirchner's regime ends in 2015, when Macri becomes President with the political coalition Cambiemos. By the time Macri takes office, the inflation rate was 27.8\% (with an upward trend) and the exchange rate was over-appreciated. The central government deficit was a 6.9\% of GDP (two times as large as it was in 2001). The real economy in stagflation since 2011. The Pontificia Universidad Católica Argentina estimated a household poverty rate of 30\%.\footnote{Because of the kirchners tampering with official statistics, there are no reliable official poverty rate estimations for this period.}

The beginning of Macri's president marks the beginning of macroeconomic populism \textit{stage 4}. In real terms, government spending decreased 20\% under Macri's 4-year presidency. The first year of Macri's presidency the Treasury's deficit raised to 9.5\% of GDP, but his last year in office ended with a deficit of 4.7\% of GDP. The economy remained stagnated and the inflation rate continued its upward trend, with an average rate of 41\% for Macri's presidential term.\footnote{During Macri's presidency the new authorities of the central bank implemented an inflation targeting regime that broke down within two years. See Cachanosky \parencite*{Cachanosky2021a} and Sturzenegger \parencite*{Sturzenegger2017, Sturzenegger2020}.} Macri's government also relied on the International Monetary Fund to secure financial help during the currency crises of 2018.

The end of Macri's presidency looks like a beginning of a new round of the 4 stages of macroeconomic populism. Macri's loses the presidential elections for a second term against Alberto Fernández, former Chief of the Cabinet of Ministers under the Kirchner's between May 2003, and August 2008. Once again an economic crisis, in this cases in the form of a currency crisis in 2018, precedes the election of a populist leader.

\begin{table}[!h]
\begin{center}
\caption{Argentina. Stages of macroeconomic populism} \label{table:ARG_Stages}
\begin{tabular}{l l l l l l} \\ \toprule
  Stages   & Start     & End       & Length             & End of stage event     & President(s)  \\ \midrule
  Stage 1  & May, 2003 & Jan, 2007 & 3 years, 8 months  & Inflation starts       & Néstor \& CFK \\
  Stage 2  & Jan, 2007 & Oct, 2011 & 4 years, 9 months  & Capital controls       & CFK           \\
  Stage 3  & Oct, 2011 & Jan, 2016 & 4 years, 3 months  & Presidential elections & CFK           \\
  Stage 4  & Jan, 2016 & Jan, 2020 & 4 years            & Return of populism     & Macri         \\ \midrule
  Total    &           &           & 12 years, 8 months &                        &               \\
  \bottomrule 
\end{tabular}
\end{center}
\end{table}

\subsubsection{The four stages of macroeconomic populism}

We propose the following 4 stages of macroeconomic populism for the argentine case. \textit{Stage 1} begins with Néstor´s presidency in 2003 and ends with the start of inflation in 2007. \textit{Stage 2} ends with the imposition of capital controls in 2011. \textit{Stage 3} ends with the beginning of Macri's presidency (2016-2020). Table \ref{table:ARG_Stages} shows the four stages of macroeconomic populism in Argentina and the following tables show the behavior of key economic variables.

In terms of real GDP (figure \ref{fig:ARG_GDP}), stage 1 benefits from a bounce-back from the 2001 crisis and the rise in the price of commodities \parencite{Ocampo2015, Thomas2015}. Growth continues in stage 2 but stagnates at the beginning of stage 3 in 2011. Inflation starts in 2007, therefore starting in 2011 Argentina is in stagflation, situation that remains through all of stage 4 (2016 - 2020) as well.

\begin{figure}[!h]
    \caption{Argentina. Real GDP (2017 prices)}
    \centering
    \includegraphics[width=\textwidth]{Images/ARG_GDP.pdf}
    \footnotesize{\textit{Source}: Penn World Table 10.0.}
    \label{fig:ARG_GDP}
\end{figure}

Real wages (figure \ref{fig:ARG_RealWage}) follow a similar pattern to real GDP with a clearer decline starting in 2017 (mid of stage 4). Real wage behavior follows Dornbusch \& Edwards \parencite*{Dornbusch1990} narrative, with the difference that they state that real wages fall to a level lower than the beginning of the populist cycle. A reason real wages do not fall to pre-populist levels is because of the impact of high commodity prices in GDP in the real wage estimation.\footnote{We estimate real wages using Penn World Table 10.0 data in the following way: Real wages $(w/P)$ equal the share of labor compensation in GDP at current national prices$(\omega)$ times real GDP at 2017 national prices $(Y)$ divided by the number of persons engaged $(N)$: $\frac{w}{P} = \frac{\omega Y}{N}$.}

\begin{figure}[!h]
    \caption{Argentina. Labor compensation}
    \centering
    \includegraphics[width=\textwidth]{Images/ARG_RealWages.pdf}
    \footnotesize{\textit{Source}: Authors' estimation based on Penn World Table 10.0.}
    \label{fig:ARG_RealWage}
\end{figure}

Government spending (figure \ref{fig:ARG_GovSpending}) also maps the four stages of macroeconomic populism, going from 13\% of GDP (2003) to 24\% of GDP at the end of the Kirchner administration (2015). The growth of government's size is steady and continues even after the economy falls into stagflation. Austerity measures, in terms of government spending over GDP, start in the second year of Macri's administration. After increasing the size of the state to 26\% in the first year, it then goes down to 22\% of GDP by 2019. The first year increase and the slow adjustment that follows is the result of the gradualist approach to shrinking the size of the government of Macri's government. Macri's gradualist approach was at odds with the central bank intention of a quick reduction of the inflation through an inflation targeting regime. The inconsistencies in the economic policy finally led to a currency crisis and return of the kirchneristas to the government in 2020, with Alberto Fernández (former Chief of Cabinet of Ministers during Néstor's presidency) as President and CFK as Vice-President \parencite{Cachanosky2021a, Sturzenegger2019}. 

\begin{figure}[!h]
    \caption{Argentina. Government Spending (\%GDP)}
    \centering
    \includegraphics[width=\textwidth]{Images/ARG_GovSpending.pdf}
    \footnotesize{\textit{Source}: European Commission for Latin America and the Caribbean (ECLAC)}
    \label{fig:ARG_GovSpending}
\end{figure}

Central bank reserves (figure \ref{fig:ARG_Reserves}) increase during stage 1 of macroeconomic populism, but stagnate through all of stage 2. Despite strict capital controls, central bank reserves start fall all of stage 3 only to increase again in stage 4. The sharp fall of reserves in 2018 is due to the currency crisis and the breakdown of the inflation targeting regime. The behavior of the central bank reserves offer a close match to the stylized 4 stages of macroeconomic populism.

\begin{figure}[!h]
    \caption{Argentina. Central bank reserves}
    \centering
    \includegraphics[width=\textwidth]{Images/ARG_Reserves.pdf}
    \footnotesize{\textit{Source}: International Monetary Fund}
    \label{fig:ARG_Reserves}
\end{figure}

In terms of inflation (figure \ref{fig:ARG_Inflation}), the series shows an upward trend that surpasses the 40\% peak of the 2001 crisis. The high inflation cycle starts in 2007, when inflation jumps from 9.8\% to 25.7\%. Nestor's average inflation rate was 15\%, CFK's was 25.6\%, and Macri's was 41\%. The year 2007 is also when the Kirchner administration starts to tamper with official inflation numbers, interference that did not stop until Macri's presidency starts in 2016.\footnote{The tampered inflation numbers are replaced with a composite of private estimations that where published as "Inflación Congreso."} The initial attempts to reduce inflation through an inflation targeting regime seems to have results, as inflation falls from 40.7\% in 2016 to 24.7\% in 2017. However, the situation soon runs out of control and the inflation rate surpasses the highest levels of the Kirchner administration.

\begin{figure}[!h]
    \caption{Argentina. Inflation}
    \centering
    \includegraphics[width=\textwidth]{Images/ARG_Inflation.pdf}
    \footnotesize{\textit{Source}: Inflación Congreso (manually collected by the authors) and Instituto de Estadísticas y Censos (INDEC)}
    \label{fig:ARG_Inflation}
\end{figure}

As explained in the previous section, the energy sector was particularly affected by the Kirchner populist policies. Argentina went from being a net export of energy to a net import by the end of stage 2 (figure \ref{fig:ARG_Energy}. This energy trade deficit became an important drain for the central bank reserves. The energy trade start to revert midway stage 4.

\begin{figure}[!h]
    \caption{Argentina. Net imports of energy (TJ)}
    \centering
    \includegraphics[width=\textwidth]{Images/ARG_Energy.pdf}
    \footnotesize{\textit{Source}: International Energy Agency}
    \label{fig:ARG_Energy}
\end{figure}

%==============================================
\subsection{Bolivia}

Evo Morales a was labor union organizer for the \textit{cocaleros} (coca leaf growers), and has led the party Movement to Socialism (\textit{Movimiento al Socialismo}, MAS) since 1998. In 2006, he was elected president of Bolivia, being the country's first president coming from an indigenous population. He remained in office for three consecutive terms, until 2019, when he resigned after an episode of electoral fraud\footnote{Morales was winning by a less than 10 p.p. margin, when reporting of electoral results was shutdown for 20 hours. When reporting was reestablished, Morales was winning with margin just over 10\%, which would avoid a runoff according to Bolivian electoral rules. The final count put Morales with 47.08\% of the votes, while the runner-up, Carlos Mesa, finished with 36.51\% of votes. See Escobari and Hoover \parencite*{Escobari2020evo}.}. 

On May 1st, 2006, Morales signed  a Supreme Decree, putting into practical effect the new Hydrocarbon Law\footnote{Technically the Law was approved by the congress one year before, on May 2005, with support and leadership provided by Evo's party, MAS. Carlos Mesa, Bolivia's president at the time, refused to sign the bill into effective, and that had to be done by the president of the Senate. Regardless, the law only went into practical effect with the publication of further provisions under Supreme Decree signed by Morales}, giving the state the "property, possession and absolute control" over the country's natural gas reserves, which constitute one-third of the country's exports revenues. Under the new decree, taxes and royalties paid by companies went up from 18 to 82\%\footnote{As the Morales administration used to say: before the multinational companies kept 82\%, and the people receive only 18\%; now it is the opposite. See Sivak \parencite*{sivak2010evo}}, As a result, revenues from hydrocarbon extraction increased from US\$173 million in 2002 to US\$1.57 billion in 2007 \parencite[p. 181]{Harten2011}. A couple of days after the publication of the decree, the Bolivian army invaded two refineries owned by Brazilian state-owned Petrobras \parencite{Maisonnave2006}\footnote{While Bolivia ended up paying US\$ 112 million for the refineries, the relationship between Morales and Petrobras during Lula's presidency in Brazil is full of controversy. For instance, Petrobras lost US\$ 434 million in an agreement with the Bolivian government signed by Lula. Against the recommendation of its engineers, Petrobras bought "rich gas" from Bolivia (a type of natural gas with heavier hydrocarbons and higher calorific value) despite the fact that the compound was totally useless for the company \parencite{Veja2015}.}. 

During his firm term, Morales had to deal with several episodes of domestic unrest. In 2006, elections for a constitutional assembly were called alongside a referendum on approval of plan to grant greater political autonomy to the regions in Bolivia. Autonomy received large support in eastern departments\footnote{Increasing regional autonomy was one of Morales' campaign promises. This was especially called by the eastern departments of Santa Cruz, Beni, Pando, and Tarija, who are among the richest of Bolivia}, but the overall population (57.6\%) rejected the proposition. Tension accumulated over the next two years, until 2008 when multiple conflicts were registered. Activists from eastern states launched strikes and protests, occupied and seized government buildings and natural gas infrastructure, led to the explosion of a gas pipeline, and armed conflicts resulting in 30 deaths\footnote{For a discussion of the 2008 conflicts, see Sivak \parencite*[p. 210-22]{sivak2010evo}}. These conflicts impacted private investments and led to some capital flight \parencite[p. 3]{Weisbrotetal2009}

In 2009 the new constitution is finally adopted, making Bolivia a Plurinational State, implementing several electoral reforms\footnote{Electoral authorities become a fourth constituted power, rules for electing members of parliament are changed, the Senate is enlarged, and the possibility of recall elections are introduced. Another import reform are the introduction of a term-limit for president, with the caveat that it does not apply to current elected official, giving Morales two extra terms.}, reforms the judiciary (judges shall be elected, not appointed), declares the coca as a national and cultural heritage, and restricts land ownership to a maximum of 5,000 hectares (12,400 acres).

Bolivia's economic performance during his initial mandate was praised by some authors as "remarkable", in part due to a "large-scale and well-timed increase in public spending", \parencite[p. 6]{Weisbrotetal2009}. This expansive fiscal policy was funded by the new natural gas revenues, which grew by a factor of nearly 7, from \$731 million to \$4.95 billion \parencite[p.1]{ArauzWeisbrotetal2009}.  Indeed, under Morales, especially after 2008, Bolivia saw a steep growth in income, reduction in poverty and unemployment. 

%==============================================
\subsection{Ecuador}

\subsubsection{The economic populist-policies}

Ecuador is an interesting left-leaning populism case due to being a dollarized economy since 2000. Because of its dollarization, Correa was unable to monetize government spending and there is no room for fiscal dominance over monetary policy \parencite[][p.34]{Edwards2019}. When Rafael Correa assumed the presidency in January 2007, Ecuador has been dollarized since year 2000. Even though dollarization did not stop a populist as strong as Correa to reform the constitution, it did impose a limit on his populist policies. Dollarization also played a role in stopping populist policies after Correa's presidential terms \parencite{Cachanosky2021c}. 

When Correa took office, Ecuador had a budget surplus of 2.13\% of GDP.\footnote{Rafael Correa was the Economy Minister under the presidency of Alfredo Palacios (2005 - 2007).} Correa soon walks into a budget deficit in 2009, leaving office with a deficit of 5.9\% of GDP. Without the possibility to monetize its deficit, Correa's administration relied on tax increases to finance his budget deficits. In 2008, Correa imposed a tax of 0.5\% on capital flows. This tax increased to 5\% in 2011, representing 10\% of government revenue in 2012. In addition to taxing capital flows, Correa also imposed a windfall tax on mining and oil industries and a tax on assets held abroad \parencite[][p. 236]{Clark2019}. Another fiscal measure was his attempt to launch what can be considered the first central bank digital currency, the \emph{dinero electrónico} \parencite{Arauz2021, Cachanosky2021c}. The launch of the \emph{dinero electrónico} failed, but it is probably no accident that Correa's attempt to issue his digital currency coincides with a significant increase in Ecuador's country risk (as measured by JP Morgan's EMBI+). In addition to tax increases, Correa also relied on debt to finance his budget deficits. In 2006, government debt in terms of GDP was 28.8\%, by 2017 it was 44.6\%. 

In 2008, Correa's administration started an antagonistic and populist rhetoric against creditors with the objective to produce a fall in the price of sovereign bonds by increasing Ecuador's perceived country risk. Then, Ecuador would buyback its sovereign bonds with financial aid from Venezuela \parencite{EcuadorCamera}. Another one of Correa's debt policies was to, after the 2008 default, to borrow from China payable with rights to oil exports in Ecuador's territory \parencite{Beittel2018}. The debt's payment conditions with China is a reason behind Correa's nationalization of the oil industry.

After its dollarization, and with the rise in commodity prices, Ecuador's economy grows consistently between 2000 and 2015 (with the exception of a slowdown during the 2008 international crisis). Ecuador's economy is stagnated since 2015. Years of increasing populist policies, fall in oil prices, and a severe earthquake in 2016 (as measured by the Modified Mercalli intensity scale) contribute to Ecuador's stagnation. In terms of GDP, government spending starts to fall in 2014 as Correa is forced to put upon himself some austerity measures.

Ecuador's stagnation (including a series of anti-Correa protests in 2015) in addition to a series of controversial regulations deteriorate Correa's political capital. Correa was not eligible for a new presidential term. In his place, he announces that Lenin Moreno, his Vice President during his first presidential term, would run for president within Correa's PAIS Alliance. Moreno is elected president in the second round of the presidential elections. Even though Moreno was expected to continue with Correa's policies, soon after taking office his domestic and international policy turns 180 degrees \parencite[see][]{Cachanosky2021c}.

We propose the following four stages of populism for Ecuador. \emph{Stage 1} begins with Correa's presidency in 2007 and ends in 2012 with stagnation in labor compensation. \emph{Stage 2} ends with the launch of the \emph{dinero electrónico} in 2014. \emph{Stage 3} ends with the 2017 presidential elections. Finally, \emph{stage 4} which starts with Moreno's turn against Correa is still ongoing.

\begin{table}[!h]
\begin{center}
\caption{Ecuador. Stages of macroeconomic populism} \label{table:ECU_Stages}
\begin{tabular}{l l l l l l} \\ \toprule
  Stages         & Start     & End       & Length & End of stage event            & President(s)  \\ \midrule
  Stage 1        & Jan, 2007 & Dec, 2012 & 5Y 11M & Labor compensation stagnation & Correa        \\
  Stage 2        & Dec, 2012 & Mar, 2014 & 1Y 3M  & Launch of digital currency    & Correa        \\
  Stage 3        & Mar, 2014 & May, 2017 & 3Y 2M  & Presidential elections        & Correa        \\
  Stage 4        & May, 2017 &           &        & Return of populism            & Moreno, Lasso \\ \midrule
  Total populism &           &           & 10Y 4M &                               &               \\
  \bottomrule 
\end{tabular}
\end{center}
\end{table}

\subsubsection{The four stages of macroeconomic populism}


\begin{figure}[!h]
    \caption{Ecuador. Real GDP 2017 prices}
    \centering
    \includegraphics[width=\textwidth]{Images/ECU_GDP.pdf}
    \footnotesize{\textit{Source}: Penn World Table 10.0}
    \label{fig:ECU_GDP}
\end{figure}



\begin{figure}[!h]
    \caption{Ecuador. Labor compensation}
    \centering
    \includegraphics[width=\textwidth]{Images/ECU_RealWages.pdf}
    \footnotesize{\textit{Source}: Authors' estimation based on Penn World Table 10.0}
    \label{fig:ECU_RealWages}
\end{figure}


\begin{figure}[!h]
    \caption{Ecuador. Government spending}
    \centering
    \includegraphics[width=\textwidth]{Images/ECU_GovSpending.pdf}
    \footnotesize{\textit{Source}: European Commission for Latin America and the Caribbean (ECLAC)}
    \label{fig:ECU_GovSpending}
\end{figure}


\begin{figure}[!h]
    \caption{Ecuador. Reserves}
    \centering
    \includegraphics[width=\textwidth]{Images/ECU_Reserves.pdf}
    \footnotesize{\textit{Source}: International Monetary Fund}
    \label{fig:ECU_Reserves}
\end{figure}


\begin{figure}[!h]
    \caption{Ecuador. Inflation}
    \centering
    \includegraphics[width=\textwidth]{Images/ECU_Inflation.pdf}
    \footnotesize{\textit{Source}: International Monetary Fund}
    \label{fig:ECU_Inflation}
\end{figure}


\begin{figure}[!h]
    \caption{Ecuador. Net imports of energy(TJ)}
    \centering
    \includegraphics[width=\textwidth]{Images/ECU_Energy.pdf}
    \footnotesize{\textit{Source}: International Energy Agency}
    \label{fig:ECU_Energy}
\end{figure}


%==============================================
\subsection{Nicaragua}

%==============================================
\subsection{Venezuela}

\subsubsection{The Bolivarian Revolution}

Venezuela is indisputably the flagship example of Latin American populism. Before getting to power, Chávez attempted a coup d'état against Carlos Andrés Pérez in 1992, and was arrest after its failure. After being pardoned two years later, he founded the Fifth Republic Movement (\textit{Movimiento V [Quinta] República}, MVR), and ran for president, winning the 1998 elections.  

Chávez assumed power in 1999 and immediately sought to promote referendum to call a constitutional assembly. At the constitutional assembly opposition held merely 6 out 125 seats \parencite[p.130]{MarcanoTyszka2007}, and the whole process, from the draft to its date effective, lasted 33 days. The "Bolivarian Constitution" promote significant changes: it changed the name of the country to \textit{Bolivarian} Republic of Venezuela, transformed the bi-cameral legislative into a unicameral one system, and greatly expanded the powers of the Executive. More importantly, it prolonged the presidential term from 5 to 6 years and allowed re-elections.

Under the new constitutional order, a mega-election had to be called. Chávez was re-elected and his supporters easily won 101 out of 165 seats for the National Assembly – in part because the opposition boycotted the election, arguing against its validity. This majority shortly after authorized Chávez to rule by decree under article 203 of the recently enacted constitution. With extraordinary powers given by the \textit{Ley Habilitante} ("enabling law") for a year, Chávez sought to implement the economic reforms in 49 decrees that would guide his socialist project for Venezuela. Among the most polemic were the Law of Hydro-carbonates, that increased taxes for multinational oil companies to 30\% and created a minimum of 51\% government participation in partially state-owned companies, effectively giving share-holder control to the government; the Fishing Law, which imposed severe restriction on commercial fishing activities, favoring small artisanal fishermen, and the Law of Land and Agrarian Development, that allowed extensive land reform by expropriation of large estates, and benefits to peasants. 

Largely because of those measures, Chávez suffered an attempted coup d'état which lasted 47 hours, until the presidential guard reclaimed the seat at the Miraflores presidential palace\footnote{Pedro Carmona, the President for the Federation of Venezuelan Chambers of Commerce claimed the seat of \textit{de facto} president. His only act was the Carmona decree, that under article 9 revoked Chávez's 49 decrees under the \textit{Ley Habilitante}}. In 2004, following a general strike in the country, a coalition of parties from both left and right led by a volunteer civil association named \textit{Súmate} gathered signatures to install a recall referendum, but 59\% of voters manifested in favor of Chávez staying\footnote{A list with the names of those favoring the referendum – known as \textit{Táscon list} – was later made public, enabling the Chávez administration to persecute and discriminate against its opponent. Referendum supporters were fired from government jobs, denied benefits and issuance of official documents.}.  

The \textit{Ley Habilitante} would be used twice again by Chávez. In 2007, for 18 months, to legislate over economic, social, territorial, scientific and defense measures, as well as control of transportation, mechanisms of popular participation and rules for government institutions. In the same year Chávez also created the Bolivarian Militias, a body of civilian militias armed by the government that should complement the Venezuelan Armed Forces. In reality, these militias are better understood in the light of the Law of Communal Councils (\textit{consejos comunales}), approved in 2006, creating mechanisms of participatory democracy, inspired by the soviet councils of USSR. Councils could opine and oversee local policies and 19,500 of them were registered. Merging with armed militias, many communal councils became \textit{colectivos}, armed groups that defend the interests of the government. %(https://es.insightcrime.org/wp-content/uploads/2018/05/Venezuela-a-Mafia-State-InSight-Crime-2018.pdf)

With a nationalized oil-industry under the control of the state-owned PDVSA, oil production stalled despite the record-high price hike from 9.5 dollars per barrel in November 1998 to \$140 in June 2008. After rising steadily during the 1990's it never again reached its peak of 3.5 million barrels a day in 1997. In fact, it did not even passed the 3 million barrels per day mark under Chávez \parencite[p. 17]{Grier2016}. Increased oil revenues that served as the "piggy-bank" for the regime were boosted by higher oil prices, not quantity\footnote{\cite{Grier2016} argue that is this incapacity to enhance production was very likely due to Chávez political use of the company, as there is no exogenous reason for the decline – to the contrary, the price shock should have led to an increment production. In his political use of the company, more than 15,000 PDSVA employees were fired after a strike at the turn of the year 2002 to 2003. Many job posts were given to political supports in order to extend its control of the company}. This large increase in oil revenues funded social programs, as PDVSA  at some point "spent twice as much on off-budget government programmes as it did on taxes, royalties and dividends" %P.G., August 27, 2012. Venezuela’s oil industry: Up in smoke. The Economist. URL http://www.economist.com/blogs/americasview/2012/08/venezuelas-oil-industry.% 
Chávez also used the company to subsidize friendly regimes by selling oil at cut-rate and gain influence over the Caribbean, accepting even bananas as payment for oil\footnote{Chávez sent 53,000 barrels a day to Cuba at cut-rate prices. Through the \textit{Petrocaribe} program, oil was exchanged by local goods such as bananas or sugar \parencite{BloombergVenezuela2019}}.

The result is astonishing: under Chávez oil GDP contracted 14.27\% relative to pre-Chávez levels in 1997\footnote{Even if we compare 1997 to 2008, when oil prices hit its highest levels, oil GDP fell 10.06\%}. Mining suffered even more, a sharp 29.87\% contraction. Over the same period (1997-2013), GDP growth was driven by communications (351.1\%), financial and insurance institutions (311.37\%), and the non-profit sector (104.91\%). Manufacturing grow an average of 0.95\% a year, water and electricity 4.07\%,	construction 3.68\%, and wholesale and maintenance services grew 4.98\%\footnote{Data is calculated by Central Bank of Venezuela and available at the National Institute of Statistics ("Instituto Nacional de Estadísticas"). It is remarkable that no data exists for the post-Chávez period}. 

One should expect that given its pro-poor rhetoric, the revolution has at least accomplished remarkable features in providing a decent living to its most vulnerable people, with improvements in education, healthcare, housing, nutrition, and social welfare programs in general. Yet, there is no evidence that this happened, even in official statistics. For instance, census data shows that illiteracy increased in 8 of the 25 states. In the two poorest states, Amazonas and Delta Amaruco, where roughly 25\% of the population is below poverty line, illiteracy rates rose 8.25 and 10.75 percentage points, respectively\footnote{See \cite{RodriguezOrtega2008} for a more extensive analysis of Chávez education program}. Health indicators also worsened. As \cite{Grier2016} show, life expectancy grew slower than would otherwise without Chávez. In a compilation of statics from 1999 to 2006, \cite{Rodriguez2008} argues that the percentage of babies born under-weighted grew from 8.4 to 9.1\%, and so did the share of households with no access to running water (7.2 to 9.4\%). Families living in houses with earthen floors increased by a factor of 2.7, from 2.5\% to 6.8\% of the population, and the country was facing a housing deficit of 1.6 million homes \parencite[]{Morales2005}. The average share of budget spent on health, education, and housing in the first eight years of Chávez is 25.12\%, nearly the same as  percent, essentially identical to the amount in the eight years preceding him, 25.08\%. Infant mortality decreased at an annual rate (3.4\%) comparable to the pre-Chávez period (3.3\%) and significantly outperformed rates of other Latin American countries such as Argentina (5.5\%) , Mexico (5.2\%), and Chile (5.2\%).

%In 2006, the Law of Communal Councils (\textit{consejos comunales}) was approved, creating mechanisms of participatory democracy. Councils could opine and oversee local policies. Many communal councils became \textit{colectivos}, armed militias that defend the interests of the government. (https://es.insightcrime.org/wp-content/uploads/2018/05/Venezuela-a-Mafia-State-InSight-Crime-2018.pdf)


%From wiki:Over 19,500[2] councils were registered throughout the country and billions of dollars have been distributed to support their efforts by the government.[3]

\subsubsection{The four stages of populism in Venezuela}

The Venezuela that Chávez encountered had a GDP per capita of \$3,900 (in current US dolars) with 49\% of the population below the national poverty line, according to official statistics\footnote{Or 9.2\% of the population under the 3.20 dollars/day poverty line, according to the World Bank. Unless otherwise noted, statistics presented in this section come from the Central Bank of Venezuela.}. The history that follows is intimately tied to oil price fluctuations, though they are not the explanation to Venezuela's fall. As he stepped into power, oil was priced at 12 dollars/barrel, and oil represented 80\% of Venezuela's exports \parencite[]{Randall2013}.

In his first year, the Venezualan economy contracted 6\%, establishing a fertile terrain for expansion – \textit{stage 1} begins. Leveraged by 183\% in oil from \$12 in 1999 to 34 dollars by November 2001, headcount poverty diminished 10 percentage points by the second semester of 2001. GDP grew 3.7\% in 2000 and 3.4\% in 2001. When oil dipped to \$19 on November 2001, the economy started to go down: imports fell 45.8\%, gross capital formation declined 52.4\%, when comparing 2003 to 2001. Despite losing more than \$7 billions trying to defend the currency, the Central Bank saw the \textit{bolívar} loose 40\% of its value in the first quarter of 2002 alone \parencite{Rodriguez2008}. At the end of 2002, a two-month national strike stopped the country. In total, GDP contracted 8.9\% in 2002 and 7.8\% in 2003. By then, 55\% of the population was again below poverty lines. 

\textit{Stage 2} starts in early 2003, when Chávez implements both currency and price controls. Under "normal" circumstances, Stage 2 might have ended much earlier, as shortages and exchange rate manipulations were already commonplace. But due the oil boom starting mid-2003, Chávez gained a (partial) second chance. To be clear, the Venezuelan economy did not break in 2008 when oil price plummeted; it was already broken much earlier, but the total collapse was postponed by the oil-boom.

\begin{table}[!h]
\begin{center}
\caption{Venezuela. Stages of macroeconomic populism} \label{table:VEN_Stages}
\begin{tabular}{l l l l l l} \\ \toprule
  Stages         & Start     & End       & Length & End of stage event     & President(s) \\ \midrule
  Stage 1        & Jan, 2001 & Jan, 2003 & 2Y  & Currency and price controls & Chávez  \\
  Stage 2        & Jan, 2003 & Jun, 2008 & 5Y 5M  & Oil crash and nationalizations   &  Chávez         \\
  Stage 3        & Jul, 2008 & - & -  & - & Chávez \& Maduro          \\
  Stage 4        &  &  &      &      &         \\ \midrule
  Total populism &           &           & 23Y 1M &                        &             \\
\bottomrule 
\end{tabular}
\end{center}
\end{table}

The greatest accomplishment argued by Chávez supporters is what follows, when poverty was reduced to 27.5\% in the first semester of 2007. However, as \cite{Rodriguez2008} points out, the reduction was driven by growth, not any policy directed to the poor\footnote{In his calculations, Venezuela saw an average reduction of one percentage point in poverty for every percentage point in per capita GDP growth, "a ratio that compares unfavorably with those of many other developing countries, for which studies tend to put the figure at around two percentage points" (p.53). Even worse, \cite{Grier2016} show that income growth significantly outperformed (a \$2500 gap) the results that should have been expected without Chávez.}. In the five year period 2004-2008, Venezuela grew an average of 10.5\% a year, as oil prices sky-rocketed from 32 dollars in January 2004 to \$140 in June 2008. When Chávez assumed, oil rents were 12\% of Venezuela's GDP; by 2005, they reached 30\%. As exports more than tripled, Venezuela was able to finance a four-fold increase in imports while doubling its international reserves. Between 2006 and 2010, Chávez nationalized large shares of various industries, including cement, communications, banks, steel, food and electricity\footnote{See \cite{Reuters2009a} for a list of nationalizations}. The last two were specially problematic. The regime had been implementing price controls since 2003, and threatened to nationalize grocery stores and arrest their owners if they did not abide by the regulated prices\footnote{See \cite{RomeroNYT2007}. The largest private company, Polar, had government officials supervising its food sales division.}. In early 2009, nationalized a food plant owned by Cargill, and ordered the army to seize rice mills \parencite[]{Reuters2009b}. After the nationalization of both production and distribution of electricity under the state-owned Corpoelec the country later faced several episodes of blackouts\footnote{Billions of dollars in no bid contracts were awarded to companies in no experience power business and indicted in money laundering, corruption, and bribery. After a decade of aging equipment without proper maintenance and under-staffed on expert engineers and technicians, the power grid collapsed. \cite{DubeCastroWSJ2009}.}.

By January 2009, oil prices had plummeted to \$40, and the real, underlying situation of the economy was revealed. From 2009 to 2018, the average GDP growth was -5.6\%, with positive exceptions only on 2011 (4.2\%), 2012 (5.6\%), and 2013 (1.3\%) as oil prices remained in the 80-110 dollars range. Despite non-negative results, the economy was deteriorating in the background. To avoid total collapse as shortages and capital flight increased, between 2009 and 2013, the Venezuelan government had to spend half of its international reserves, increasing imports by nearly 40\%. 

\textit{Stage 3} begun mid-2014, when oil prices starting dropping from \$100, reaching 46 dollars by January 2015. With this new shock, the internal economy collapse. In 2014 alone, manufacturing shrunk 15\%. In 2016, GDP fell 17 percent. By 2018, the last year with available data, Venezuelan manufacturing had been reduced to one fourth of its size in 2008. Some industries such as automobiles (-95\%), tobacco (-94\%), plastic (-93\%), and metallic objects for industrial use (-93\%) nearly disappeared. 

While the real economy has seen ups and downs during Chávez, with oil price shocks as triggers, monetary policy was a constant: expansion. From 1999 to 2013, when Chávez died, M2 money supply was multiplied by a factor of 114, an average annual increase of 39.5\%. Under Nicolás Maduro, it grew 339,384,160,559 times. Cumulatively, Venezuela's populist experience increased the country M2 by more than 46.2 \textit{trillion} times. 

%https://foreignpolicy.com/2015/05/07/dont-blame-it-on-the-oil-venezuela-caracas-maduro/


% ================================================================
% --- SECTION 4: DISCUSSION
\section{Discussion} 
    \label{sec:Discussion}

% ================================================================
% --- SECTION 5: APPENDIX
\section{Appendix} 
    \label{sec:Appendix}

\begin{table}[!h]
\begin{center}
\caption{Data sources} \label{table:Sources}
\begin{tabular}{l l}  \\ \toprule
  Variable              & Source                           \\ \midrule
  Real GDP              & PWT 10.0                         \\
  Labor compensation    & Authors estimation with PWT 10.0 \\
  Gov. spending (\%GDP) & ECLAC                            \\
  Central bank reserves & IMF                               \\
  Inflation             & INDEC, Inflación Congreso, WDI   \\ 
  Net imports of energy & IEA                              \\
  \bottomrule 
\end{tabular}
\end{center}
\end{table}


% ================================================================
% --- SECTION 5: REFERENCES
\newpage

\singlespacing
\printbibliography

% ================================================================
% --- THE END
\end{document}